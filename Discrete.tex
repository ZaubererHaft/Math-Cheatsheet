\section{Logik}

\section{Grundlagen}

\begin{definition}[Menge]
	Eine Menge ist eine Zusammenfassung von mathematischen Objekten ohne Beachtung der Reihenfolge und Vielfachheiten. Wir schreiben $\{1, 2, 3\}$, was extensionale Darstellung genannt wird oder $\{i \in \naturalset \medspace | \medspace 1 \leq i \leq 3  \}$, was inentsionale Darstellung heißt. Für die leere Menge schreibt man $\emptyset := \{\}$. Objekte $x$ in einer Menge $A$ heißen Elemente und man schriebt $x \in A$
	
	Auf Mengen definieren wir die folgenden Operationen:
	\begin{itemize}[noitemsep]
		\item Teilmenge: $A \subseteq B$, falls $\forall a \in A : a \in B$
		\item Gleichheit: $A = B \leftrightarrow A \subseteq B \subseteq A$
		\item Schnitt: $A \cap B := \{ a \in A \land a \in B\}$
		\item Vereinigung: $A \cup B := \{ a \in A \lor b \in B\}$
		\item Symmetrische Differenz: $A \Delta B := \{a \in A \oplus b \in B\}$
		\item Schnitt von Mengen von Mengen $\bigcap_{M \in S} M := \{x \medspace | \medspace \forallin{M}{S}: x \in M\}$
		\item Vereinigung von Mengen von Mengen $\bigcup_{M \in S} M := \{x \medspace | \medspace \exists M \in S: x \in M\}$
	\end{itemize}
\end{definition}

\begin{definition}[Tupel, Sequenz, Folge]
	Unter einem Tupel versteht man die Zusammenfassung von Objekten unter Beachtung der Reihenfolge und Vielfachheiten
\end{definition}

\begin{definition}[Relationen]
	content...
\end{definition}

\begin{definition}[Funktionen]
	content...
\end{definition}

\begin{definition}[Kardinalitäten]
	content...
\end{definition}

\begin{definition}[Multimenge]
	content...
\end{definition}

\section{Graphentheorie}

\section{Beweise}

\section{Kombinatorik}

\section{Zahlentheorie}

