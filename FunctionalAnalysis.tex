\section{Funktionalanalysis}

\begin{definition}[Beschränkte Funktion]
	Sei $ \Omega \neq \emptyset \subseteq \realset^n, V = \aspace{\Omega}{\realset}$, Menge aller Funktionen von $\Omega$ nach $\realset$. $f \in V$ wird beschränkt genannt, falls $\exists c \in \realset : \forallin{x}{\Omega} : |f(x)| < c$. Die Menge aller Beschränkten Funktionen  $\bspace{\Omega}{\realset}$ ist ein Unterraum von $V$.
\end{definition}

\begin{definition}[Offene (Punkt-) Menge]
	Sei $x \in \realset^n, \epsilon > 0$. $B_\epsilon(x) := \{y \in \realset^n \medspace | \medspace \norm{y-x}_2 < \epsilon \}$ heißt offener Kugelkörper mit Radius $\epsilon$. Eine Menge $\Omega \subseteq \realset^n$ heißt offen in $\realset^n$, falls $\forallin{x}{\Omega} : \exists \epsilon > 0 : B_\epsilon(x) \subseteq \Omega$
\end{definition}

\begin{definition}[Geschlossene (Punkt-) Menge]
	Eine Menge $A \subseteq \realset^n$ heißt geschlossen, falls $\realset^n \setminus A = \{x \in \realset^n \medspace | \medspace x \notin A \}$ offen ist
\end{definition}

\begin{definition}[Raum stetig differenzierbarer Funktionen]
	Sei $\Omega \subseteq \realset^n, \Omega$ nicht leer und offen, $k \in \naturalset$
	\begin{itemize}[noitemsep]
		\item $\cnspace{k}{\Omega}{\realset}$ bezeichnet den Raum der kontinuierlich differenzierbaren Funktionen, das heißt für alle $f \in \cnspace{k}{\Omega}{\realset}$ existiert die Ableitung bis zur Ordnung $k$ und ist stetig
		\item $\cnspace{0}{\Omega}{\realset}$ bezeichnet den Raum der stetigen Funktionen
		\item $\cnspace{\infty}{\Omega}{\realset}$ bezeichnet den Raum der unendlich oft stetig differenzierbaren Funktionen
	\end{itemize}
\end{definition}

\begin{definition}[Funktional]
	Sei $V$ ein Vektorraum über den Körper $\fieldk$. Ein Funktional $\Phi$ ist eine Abbildung $\Phi : V \rightarrow \fieldk$, also eine Abbildung von einen Vektorraum in den zugehörigen Körper. Ist ein Funktional linear, so heißt es Linearform.
\end{definition}


\begin{definition}[Bilinearform / Bilineares Funktional]
	Eine Abbildung $\function{a}{V \times W}{\realset}$ zwischen zwi Vektorräumen $V,W$ wird Bilineaform genannt, falls $a$ linear in jedem Argument ist, d.h.
	\begin{itemize}[noitemsep]
		\item $\forallin{v_1, v_2}{V}, \forallin{w}{W}: a(v_1 + v_2, w) = a(v_1,w) + a(v_2,w)$
		\item $\forallin{v}{V}, \forallin{w_1, w_2}{W}: a(v, w_1 + w_2) = a(v,w_1) + a(v,w_2)$	
		\item $\forallin{\lambda}{\fieldk}, \forallin{v}{V}, \forallin{w}{W}: a(\lambda v, w) = \lambda a(v,w) + a(v,w)$	
		\item $\forallin{\lambda}{\fieldk}, \forallin{v}{V}, \forallin{w}{W}: a( v, \lambda w) = a(v,w) + \lambda a(v,w)$	
	\end{itemize}
	Man schreibt $\inner{u}{v} := a(u,v)$
\end{definition}

\begin{definition}[Skalarpodukt]
	Sei $V$ ein Vektorraum über $\fieldk$. Eine Bilinearform $\function{a}{V \times V}{\realset}$ heißt Skalarprodukt, falls
	\begin{enumerate}[noitemsep]
		\item $\forallin{u,v}{V} : a(u,v) = a(v,u)$ (Symmetrie)
		\item $\forallin{v \neq 0}{V} : a(v,v) > 0$ (Positivität)
		\item $\forallin{v}{V} : a(v,v) = 0 \leftrightarrow v = 0 $ (Eindeutigkeit des neutralen Elements)	
	\end{enumerate}
\end{definition}

\begin{definition}[Norm einer Funktion]
	Sei $V$ ein Vektorraum über $\fieldk$. Eine Abbilduhg $\function{\normempty}{V}{\realset}$ heißt Norm auf $V$, falls $\forallin{u,v}{V}, \alpha \in \realset$ gilt:
	\begin{enumerate}[noitemsep]
		\item $\norm{v} > 0, \norm{v} = 0 \leftrightarrow v = 0$ (Positivität)
		\item $\norm{\alpha v} = |\alpha| \cdot \norm{v}$ (Absolute Homogenität)
		\item $\norm{u + v} \leq \norm{u} + \norm{v}$ (Dreiecksungleichung)
	\end{enumerate}
 \end{definition}

\begin{definition}[Induzierte/Natürliche Norm]
	Sei $V$ ein Vektorraum mit Skalarprodukt $\innerempty$. Dann definiert $\norm{v} := |\inner{v}{v}|^\frac{1}{2}$ eine Norm auf $V$, genannt natürliche Norm oder Norm induziert von $\innerempty$.
\end{definition}

\begin{satz}[Cauchy-Schwarz Ungleichung]
	Sei $V$ ein reller Vektorraum mit Skalarprodukt $\innerempty$. Dann gilt: $|\inner{u}{v}|^2 \leq \inner{u}{u} \inner{v}{v}$. Ist $\normempty$ die induzierte norm, dann gilt: $|\inner{u}{v}| \leq \norm{u} \cdot \norm{v}$
\end{satz}

\begin{definition}[Konvergenz auf Vektorräumen]
	Sei $V$ ein reller Vektorraum mit Norm $\normempty : V \rightarrow \realset$, sei $\sequence{v}{n}$ Folge in $V$, d.h. $\forallin{n}{\naturalset} : v_n \in V$. $\sequence{v}{n}$ heißt konvergent, falls $\exists v \in V: \liminfty{n} \norm{v_n - v} = 0$ bzw. $\forall \epsilon > 0 : \exists N \in \naturalset : \forall n \geq N : \norm{v_n -v} < \epsilon$. $v$ heißt dann Grenzwert der Folge. Man schreibt $v = \liminfty{n} v_n$ oder $v_n \overset{n \rightarrow \infty}{v}$.
\end{definition}

\begin{satz}[Eindeutigkeit des Grenzwerts]
	Falls ein Grenzwert zu einer Folge existiert, so ist es eindeutig. Für endlich-dimensionale Vektorräume konvergiert eine Folge mit einer Norm $\normempty_a$ genau dann, wenn es mit einer anderen Norm $\normempty_b$ konvergiert aufgrund der Äquivalenz von Normen auf endlich-dimensionalen Vektorräumen. Für unendlich-dimensionale Vektorräume gilt dies im Allgemeinen nicht.
\end{satz}

\begin{definition}[Punktweise Konvergenz]
	Sei $\Omega \neq \emptyset \subseteq \realset^n$. Sei $\sequence{f}{n}$ eine Folge an Funktionen, $f_n \in \aspace{\Omega}{\realset}$. Die Folge heißt punktweise konvergent, falls $\forallin{x}{\Omega}$ die Folge $(f_n(x))_{n \in \naturalset}$ konvergent in $\realset$ ist. Sei $f \in \aspace{\Omega}{\realset}$ mit $f(x) = \liminfty{n}f_n(x)$. $f$ heißt dann punktweiser Grenzwert von $\sequence{f}{n}$
\end{definition}

\begin{definition}[Gleichmäßige Konvergenz]
	Sei $\Omega \neq \emptyset \subseteq \realset^n$, $f \in \bspace{\Omega}{\realset}$ mit Norm $\norm{f}_\infty := sup_{x \in \Omega}|f(x)|$. Sei $\sequence{f}{n} \subseteq \bspace{\Omega}{\realset}$. Falls $f_n \overset{n \rightarrow \infty}{\rightarrow} f$ bzgl. der Norm $\normempty_\infty$, so konvergiert $f_n$ gleichmäßig gegen $f$.
\end{definition}

\begin{satz}[Eigenschaften gleichmäßiger Konvergenz]
	Konvergiere $f_n \rightarrow f$ gleichmäßig. Dann gilt:
	\begin{itemize}[noitemsep]
		\item $(f_n(x))_{n \in \naturalset}$ konvergiert punktweise gegen $f$
		\item Falls alle $f_n$ stetig sind, so ist der Grenzwert $f$ auch stetig 
	\end{itemize} 
\end{satz}

\begin{definition}[$\epsilon$ - $\delta$ Kriterium für Stetigkeit ]
	Seien $V,W$ relle Vektorräume mit Normen $\normempty_V, \normempty_W$. Sei $\Omega \subseteq V$. $\function{f}{\Omega}{W}$ heißt stetig im Punkt $v_0 \in \Omega$, falls $\forall \epsilon > 0: \exists \delta > 0: \forallin{v}{V} : \norm{v - v_0}_V < \delta \rightarrow \norm{f(v) - f(v_0}_W < \epsilon$.
\end{definition}

\begin{definition}[Folgenkriterium für Stetigkeit ]
	Seien $V,W$ relle Vektorräume mit Normen $\normempty_V, \normempty_W$. Sei $\Omega \subseteq V$. $f$ heißt stetig im Punkt $v_0 \in \Omega$, falls für jede Folge $\sequence{v}{n}$ in $\Omega$ mit $vn \overset{n \rightarrow \infty}{\rightarrow} v_0$ gilt dass $f(vn) \overset{n \rightarrow \infty}{\rightarrow} f(v_0)$, bzw. $\liminfty{n} \norm{v_n - v_0}_V = 0 \Rightarrow \liminfty{n} \norm{f(v_n) - f(v_0)}_W = 0$ 
\end{definition}

\begin{definition}[Beschränkte Lineare Funktion]
	Seien $V,W$ relle Vektorräume mit Normen $\normempty_V, \normempty_W$. Eine lineare Abbildung $\function{L}{V}{W}$ heißt beschränkt, falls $\exists C > 0 : \forallin{v}{V} : \norm{L(v)}_W \leq C \cdot \norm{v}_V$.
\end{definition}

\begin{satz}[Stetigkeit Linearer Funktionen]
		Seien $V,W$ relle Vektorräume. Eine lineare Abbildung $\function{L}{V}{W}$ ist genau dann stetig, wenn
		 $L$ stetig in $v_0 = 0$ ist bzw. genau dann, wenn $L$ beschränkt ist.
\end{satz}

\begin{definition}[Cauchy Folge]
	Sei $V$ reller Vektorraum, $\normempty_V$ Norm auf $V$. Eine Folge $\sequence{v}{n}$ in $V$ heißt Cauchy-Folge, falls $\forall \epsilon > 0 : \exists N \in \naturalset : \forall n,m \geq N : \norm{v_n - v_m}_V < \epsilon$. Falls $\sequence{v}{n}$ konvergent ist, dann ist $\sequence{v}{n}$ Cauchy-Folge.
\end{definition}

\begin{definition}[Vollständiger Vektorraum]
		Sei $V$ reller Vektorraum, $\normempty_V$ Norm auf $V$. $V$ heißt vollständig, falls jede Cauchy Folge $\sequence{v}{n}$ in $V$ konvergiert bzgl. der Norm $\normempty_V$. 
\end{definition}

\begin{definition}[Banach-Raum]
	Falls ein Vektorraum $V$ zusammen mit einer Norm $\normempty$ vollständig ist, so heißt $V$ Banach-Raum. 
\end{definition}

\begin{definition}[Hilbert-Raum]
	Sei $V$ Vektorraum mit $\innerempty $ Skalarprodukt. Sei $\norm{v} := \sqrt{\inner{v}{v}}$ Norm. Falls $V$ Banach-Raum bzgl. $\normempty$ ist, dann heißt $V$ Hilbert-Raum bzgl. $\innerempty$.
\end{definition}

\begin{definition}[Träger]
	Der Träger einer Funktion $f$ umfasst die Menge aller $x$ sodass $f(x) \neq 0$ : $supp(f) = \{ x \medspace | \medspace f(x) \neq 0 \}$. Der Träger einer Funktion $f \in \cnspace{\infty}{[a,b]}{\realset}$ ist definiert als $supp(f) = \{x \medspace | \medspace \exists(x_n) \medspace \text{mit} \medspace x_n \rightarrow x \medspace \text{und} \medspace f(x_n) \neq 0 \}$. Der Träger heißt kompakt, falls er geschlossen und beschränkt ist.
\end{definition}

\begin{definition}[Testfunktionenraum]
	Der Testfunktionenraum kann definiert werden als
	\begin{itemize}[noitemsep]
		\item $\testspace{[a,b]}{\realset} := \{ \function{\phi}{[a,b]}{\realset} \medspace | \medspace \phi \in \cnspace{1}{[a,b]}{\realset} \land \phi(a) = \phi(b) = 0 \}$
		\item $\testspacecompact{(a,b)}{\realset} := \{ \phi \in \cnspace{\infty}{[a,b]}{\realset} \medspace | \medspace \phi \medspace$ hat kompakten Träger, der in $(a,b)$ enthalten ist.  $\}$.
	\end{itemize}
\end{definition}

\begin{definition}[$L^p$ - Raum]
	Der $L^p$-, oder auch Lebesgue-, Raum ist definiert als: $L^p([a,b], \realset) := \{ f \in \aspace{[a,b]}{\realset} \medspace | \medspace f $ ist messbar und $ \medspace \int_a^b |f(x|^p \medspace d\lambda < \infty \}, p \in \naturalset$. Jeder $\lpspace{p}{[a,b]}{\realset}$-Raum ist ein Banach-Raum mit der Norm: $\norm{f}_{L^p} := (\int_{a}^{b} |f(x|^p \medspace d \lambda )^\frac{1}{p}$.
\end{definition}

\begin{satz}[$L^2$ - Hilbert-Raum]
	Der $\lpspace{2}{[a,b]}{\realset}$ ist der einzige Hilbert-Raum unter den $L^p$-Räumen mit dem Skalarprodukt $\inner{f}{g} := \int_{a}^{b} |f(x)g(x)| d\lambda$
\end{satz}

\begin{definition}[Schwache Ableitung]
	Sei $f \in \lpspace{2}{[a,b]}{\realset}$. $g \in \lpspace{2}{[a,b]}{\realset}$ heißt Schwache Ableitung, falls $\forallin{\phi}{\testspace{[a,b]}{\realset}}$ gilt:
	\begin{align*}
		\int_{a}^{b} f(x) \phi'(x) dx = - \int_{a}^{b} g(x) \phi(x) dx
	\end{align*}
	Falls eine klassische/starke Ableitung $f'(x)$ existiert, so ist sie gleich der schwachen Ableitung $g$. 
\end{definition}

\begin{definition}[Abgeschlossene Hülle]
	Sei $V$ reller Vektorraum mit Norm $\normempty$. Sei $A \subseteq V$. Die abgeschlossene Hülle von $A, \bar{A}$ ist die Menge aller $v \in V$ sodass es eine Folge $\sequence{a}{n}$ mit $a_n \in A $ mit $\liminfty{n} a_n = v$.
\end{definition}

\begin{satz}[Dichte der Testfunktionenräume]
	Der Raum $\testspacecompact{(a,b)}{\realset}$ is dicht in $\lpspace{2}{[a,b]}{\realset}$, d.h., $\forallin{f}{\lpspace{2}{[a,b]}{\realset}} : \exists \sequence{\phi}{n}$ mit $\phi_n \in \testspacecompact{(a,b)}{\realset}$ sodass $\phi_n \overset{n \rightarrow \infty}{\rightarrow} f$ bzgl. $\normempty_{L^2}$. Alternativ kann dies mittels der abgeschlossenen Hülle ausgedrückt werden:
	\begin{align*}
		\overline{\testspacecompact{(a,b)}{\realset}} = \lpspace{2}{[a,b]}{\realset} \Rightarrow \overline{\testspace{[a,b]}{\realset}} = \lpspace{2}{[a,b]}{\realset}
	\end{align*}
\end{satz}

\begin{satz}[Eindeutigkeit der Schwachen Ableitung]
	Die schwache Ableitung ist eindeutig als Element von $\lpspace{2}{[a,b]}{\realset}$.
\end{satz}

\begin{definition}[Schwache Ableitungen höherer Ordnung]
	Sei $f \in \lpspace{2}{[a,b]}{\realset}$ mit $f' \in \lpspace{2}{[a,b]}{\realset}$ schwacher Ableitung. Falls $f'$ wieder eine schwache Ableitung $h \in \lpspace{2}{[a,b]}{\realset}$ besitzt, schreibt man $f'' := h$. In gleicher Art und weise sind höhere schwächere Ableitungen $f^{(n)}$ definiert.
\end{definition}


\begin{definition}[Sobolev-Raum]
	Der Sobolev-Raum der Ordnung $m$ ist definiert als $\sobolevspace{m}{[a,b]}{\realset} := \{ f \in \lpspace{2}{[a,b]}{\realset} \medspace | \medspace f^{(k)} \in \lpspace{2}{[a,b]}{\realset} $ (schwache Ableitung) existiert für $ k=1,...,m \} $.
\end{definition} 

\begin{satz}[Sobolev Hilbert-Raum]
	Seien $u,v \in \sobolevspace{m}{[a,b]}{\realset}$.
	\begin{align*}
		\inner{u}{v}_{H^m} := \int_{a}^{b} u(x)v(x) + u'(x)v'(x) + ... + u^{(m)}(x)v^{(m)}(x) dx
	\end{align*} definiert ein Skalarprodukt auf $\sobolevspace{m}{[a,b]}{\realset}$. Zusammen mit der induzierten Norm $\norm{u}_{H^m} := \sqrt{\inner{u}{u}_{H^m}}$ ist der Sobolev-Raum vollständig, d.h., $\sobolevspace{m}{[a,b]}{\realset}$ formt einen Hilbert-Raum mit $\normempty_{H^m}$.
\end{satz}

\begin{definition}[Sobolev-Raum mit kompaktem Träger]
	Sei $k \geq 1$. \\ $\sobolevspacecompact{k}{(a,b)}{\realset} := \overline{\testspacecompact{(a,b)}{\realset}}$, d.h. $\sobolevspacecompact{k}{(a,b)}{\realset}$ beinhaltet alle Funktionen $f \in \sobolevspace{k}{[a,b]}{\realset}$ für die es eine Folge $\sequence{f}{n}, f_n \in \testspacecompact{(a,b)}{\realset}$ gibt, sodass $f_n \overset{n \rightarrow \infty}{\rightarrow} f$ bzgl. $\normempty_{H^k}$. $\sobolevspacecompact{k}{(a,b)}{\realset}$ formt einen Hilbertraum mit $\normempty_{H^k}$.
\end{definition}

\begin{satz}[Berechnung der schwachen Ableitung]
	Sei $-\infty < a < b < \infty$, $\function{u}{[a,b]}{\realset}$ definiere die folgenden Eigenschaften:
	\begin{itemize}[noitemsep]
		\item $u$ ist stetig auf $[a,b]$
		\item es gibt eine endliche Anzahl an Punkten $a_j, j \in \{0,...,n\}$ mit $a = a_0 < a_1 < ... < a_n = b$ sodass $u$ stetig differenzierbar für alle $j$ auf dem offenen Unterintervall $(a_j,a_{j+1})$ mit klassischer Ableitung $u'(x)$  und die Ableitung stetig fortsetzbar auf das geschlossene Intervall $[a_j, a_{j+1}]$ ist
	\end{itemize} Dann gilt: $u$ ist schwach differenzierbar mit der schwachen Ableitung
	\begin{align*}
		h(x) = \begin{cases}
		 u'(x), & \text{falls} \medspace $u$ \medspace \text{klassisch differenzierbar in } \medspace x \medspace \text{ist} \\
		 \text{beliebig} & \text{sonst}
		\end{cases}
	\end{align*}
	und $u \in \sobolevspacecompact{1}{[a,b]}{\realset}$ sowie  $u \in \sobolevspacecompact{1}{[a,b]}{\realset} \leftrightarrow u(a) = u(b) = 0$
\end{satz}

\begin{satz}[Vereinfachte Poincaré-Friedrichs-Ungleichung]
	Sei $v \in \sobolevspacecompact{1}{(0,1)}{\realset}$. Dann gilt: $\int_{0}^{1} v(x)^2 dx \leq \frac{1}{8} \int_{0}^{1} v'(x)^2 dx$
\end{satz}

\begin{satz}[Variationelle Gleichung]
	Sei $V$ ein reller normierter Vektorraum. Sei $\function{a}{V \times V}{\realset}$ bilinear, symmetrisch und positiv definit, d.h., ein Skalarprodukt. Sei $\function{L}{V}{\realset}$ eine lineare Abbildung. Definiere $\function{F}{V}{\realset} : v \mapsto \frac{1}{2}a(v,v) - Lv$. Dann gilt: $v_1 \in V$ minimiert $F$ genau dann, wenn $v_1$ die variationelle Gleichung $a(v_1, v) = Lv$ für alle $v \in V$ löst, d.h., die Abbildung $v \mapsto Lv$ ist gleich der Abbildung $v \mapsto a(v_1,v)$ 
\end{satz}

\begin{satz}[Riesz'scher Darstellungssatz]
	Sei $V$ Hilbert-Raum mit $\innerempty$. Sei $\function{L}{V}{\realset}$ lineare Abbildung und stetig bzgl der induzierten Norm $\norm{v} = \sqrt{\inner{v}{v}}$. Dann gibt es genau ein $\tilde{v} \in V$ sodass $Lv = \inner{\tilde{v}}{v}$.
\end{satz}

\begin{definition}[Koerzitive Funktion]
	Sei $V$ reller Vektorraum mit Norm $\normempty$. Eine bilineare Abbildung wird koerzitiv genannt, falls $\exists C > 0 : \forallin{v}{V} : |a(v,v)| \geq C \norm{v}^2$
\end{definition}

\begin{satz}[Lax-Milgram]
	Sei $V$ Hilbert-Raum. Sei $a$ eine stetige Bilinearform sodass $a(v,v) \geq C \norm{v}^2$ (koerzitiv). Sei $\function{L}{V}{\realset}$ linear und stetig. Dann gibt es genau ein $v_1$ sodass $Lv = a(v_1,v)$. Die Lösung der variationellen Gleichung ist damit eindeutig bestimmt.
\end{satz}

\begin{satz}[Schätzung der Lösung einer variationellen Gleichung]
	Sei $v_1$ Lösung der variationellen Gleichung $a(v_1,v) = Lv, \forallin{v}{V}$. Dann gilt: $\norm{v_1} \leq \frac{2D}{C}$, wobei $L$ beschränkt mit $|Lv| \leq D \norm{v}$ und $a$ koerzitiv mit $a(v,v) \geq C \norm{v}^2 $
\end{satz}

\begin{satz}[Anwendung auf Dirichlet Randwertprobleme]
	Sei $\Omega$ Gebiet mit Lipschitz-stetigem Rand. Betrachte die lineare elliptische partielle Differentialgleichung zweiter Ordnung
	\begin{align*}
		- \sum_{i,k = 1}^{n} \partialf{}{x_i}(a_{i,k}(x) \partialf{u(x)}{x_k}) + a_0(x)u(x) = f(x) \\
		u(x) = 0, \forallin{x}{\partial \Omega}, f \in \lpspace{2}{\Omega}{\realset}
	\end{align*}
	$u \in \cnspace{2}{\Omega}{\realset}$ heißt starke Lösung der Differentialgleichung.
	Multipliziere mit $v \in \testspacecompact{\Omega}{\realset}$ und integriere über $\Omega$. Die folgende Gleichung ist als schwache Form der Differentialgleichung definiert:
	\begin{align*}
		- \int_{\Omega} \sum_{i,k = 1}^{n} \partialf{}{x_i}(a_{i,k}(x) \partialf{u(x)}{x_k}) v(x) + a_0(x)u(x) v(x) dx = \int_\Omega f(x) v(x) dx
	\end{align*}
	Definiere nun:
	\begin{align*}
		&a(u,v) := \int_{\Omega} \sum_{i,k = 1}^{n} \partialf{}{x_i}(a_{i,k}(x) \partialf{u(x)}{x_k}) v(x) + a_0(x)u(x) v(x) dx \\
		&L(v) := \int_\Omega f(x)v(x) dx
	\end{align*}
	Dann heißt $u \in \sobolevspacecompact{1}{\Omega}{\realset}$ schwache Lösung  der Differentialgleichung genau dann, wenn $a(u,v) = L(v)$. Ist Lax-Milgram anwendbar, so ist $u$ eindeutig bestimmt und minimiert $F(v) = \frac{1}{2} a(v,v) - Lv$
\end{satz}

\begin{definition}[Galerkin-Schema]
	Eine Folge endlich-dimensionaler Unterräume $\sequence{V}{m}$ eines unendlich-dimensionalen Vektorraums $V$ heißt Galerkin-Schema, falls es für alle $v \in V$ eine Folge $\sequence{v}{m}, v_m \in V_m $ gibt, sodass $v_m \rightarrow v$ für $m \rightarrow \infty$. 
	
	Falls $\sequence{v}{j}, v_j \in V$ sodass für alle $m \in \naturalset$, $\{v_q,...,v_m\}$ linear unabhägig und $V_0 = \{\sum_{i = 1}^m a_j v_j \medspace | \medspace a_j \in \realset, m \in \natural\}$ dicht in $V$ ist ($\bar{v_0} = V$), dann heißt $(v_j)$ Galerkin-Basis von $V$.
\end{definition}

\begin{satz}[Approximation der variationallen Gleichung]
 Sei $W$ endlich dimensionaler Unterraum von $V$. Die Lösung $v_m \in W $ sodass für alle $v \in W$ gilt $a(v_m,v) = Lv$, ist eindeutig aufgrund von Lax-Milgram.
 Die Lösung kann ermitteln werden wie folgt:
 
 \begin{align*}
 	w &= \langle w_1,...,w_d \rangle, v = \sum_{i=1}^{d} \alpha_i w_i, v_w = \sum_{i=1}^{d} \zeta_i w_i \\
 	\rightarrow L_W v &= \sum_{j=1}^{d} \alpha_j L w_j = (Lw_1,...,Lw_d) \left(\begin{smallmatrix}
 		\alpha_1 \\
 		\vdots \\
 		\alpha_d
 	\end{smallmatrix}\right) =: l_w^T \alpha \\
 	a_W(v_w,v) &= a_W(\sum_{i=1}^{d} \zeta_i w_i, \sum_{j=1}^{d} \alpha_j w_j) \overset{\textbf{bilinearity}}{=} \\ &= \left(\begin{smallmatrix}
 		\zeta_1,...,\zeta_d
 	\end{smallmatrix}\right)  \left(\begin{smallmatrix}
	 		a_W(w_1,w_1) & \hdots & a_W(w_d,w_1) \\
	 		\vdots & & \\
	 		a_W(w_d,w_1) & \hdots & a_W(w_d,w_d) 
	\end{smallmatrix}\right)
	\left(\begin{smallmatrix}
		\alpha_1 \\
		\vdots \\
		\alpha_d
	\end{smallmatrix}\right) =: \zeta^T A \alpha \\
	\rightarrow L_Wv &= a_W(v_w,v) \leftrightarrow \zeta^T A \alpha = l_W^T \alpha \leftrightarrow (\zeta A - l_W^T) \alpha = 0 \\ \leftarrow \zeta A &= l_W^T \leftrightarrow A^T \zeta = l_W \leftrightarrow \zeta = (A^T)^{-1} l_W \\
	\rightarrow v_W &= \sum_{i=1}^{d} \zeta_i w_i \medspace \text{solution}
 \end{align*}
\end{satz}

\begin{satz}[Lemma von Céa]
	Sei $V$ Hilbert-Raum, $\innerempty$ und Norm $\norm{u} = \sqrt{\inner{u}{u}}$. Sei $W$ endlich dimensionaler Unterraum von $V$. Sei $\function{a}{V \times V}{\realset}$ Bilinearform mit
	\begin{itemize}[noitemsep]
		\item $|a(w,v)| \leq C \norm{w} \cdot \norm{v}, \forallin{v,w}{V}$ (Stetig)
		\item $|a(v,v)| \geq D \norm{v}^2$ (Koerziv) 
	\end{itemize}
	Dann gilt:
	\begin{align*}
		\norm{v_1 - v_W} \leq \frac{C}{D}\cdot inf_{v \in V} \norm{v_1 - v}
	\end{align*}
	und falls $a$ zusätzlich symmetrisch ist:
	\begin{align*}
	\norm{v_1 - v_W} \leq \sqrt{\frac{C}{D}} \cdot inf_{v \in V} \norm{v_1 - v}
\end{align*}
	Darauß folgt dass die Approximation der variationellen Gleichung konvergiert für eine Galerkin-Basis.
\end{satz}

\begin{satz}[Fundamentallemma der Variationsrechnung]
	Sei $G \in \cnspace{0}{[a,b]}{\realset}, $. Falls 
	\begin{align*}
		\forall \eta \in \cnspace{1}{[a,b]}{\realset}, \eta(a) = \eta (b) = 0 :  \int_{a}^{b} \eta(x) G(x) dx = 0 
	\end{align*}
	So folgt dass $G(x) = 0$.
\end{satz}

\begin{definition}[Eigenfunktionen]
	content...
\end{definition}

\begin{definition}[Faltung]
	content...
\end{definition}


\begin{definition}[Distribution]
	content...
\end{definition}

\pagebreak