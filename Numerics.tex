\section{Gaussian Elimination}

\begin{definition}[Algorithm]
	Let be $A \in \realset^{\mtimesn}, b \in \realset^n$ 
	\begin{enumerate} [noitemsep]
		\item Forward Elimination $A, b$ are transformed synchronously until $A$ is reduced to an upper triangular matrix $U$ by \begin{itemize}[noitemsep]
			\item Multiplying the $i$-th row by a nonzero scalar
			\item Add the $\alpha \in \realset$ - multiple of the $i$-th row to the $j$-th one
			\item Swapping the position of the $i$-th and $j$-th row
		\end{itemize}
		\item Backward Substitution: Using the created row echolon form $Ux = c$ the structure is used to solve the system
	\end{enumerate}
	Details: Completion of the $j$-th row is called the $j$-th principal step. $a_{j,j} \neq 0$ is called the pivot element, swap rows if $a_{j,j} = 0$. Rank and determinant of $A$ is (almost) directly obtainable
\end{definition}

\begin{definition}[Elementary Transformation Matrices]
	Assigning matrices to elementary operations for analyzing purposes
	\begin{enumerate}[noitemsep]
		\item \begin{multicols}{2}
			\begin{equation*}
				S(\alpha;i) := diag(\suboneton{d})
			\end{equation*}
			\begin{itemize}[noitemsep]
				\item $S^{-1} := diag(\frac{1}{\alpha_1}, \dots, \frac{1}{\alpha_n})$
				\item $S=S^T, S^TS = SS^t \rightarrow $ normal matrix
				\item $det(S) = \alpha, S^{-1} \neq S^T$ (non-orthogonal)
				\item Multiplying $A$ with $S$ scales a row/col with $\alpha$
			\end{itemize}
		\end{multicols}
		\item \begin{multicols}{2}
			$P(i, j)$ differs from $I_n$ in 4 elements \begin{align*}
			P(i, j)_{i, i} &= 0, P(i, j)_{i, j} = 1 \\
			P(i, j)_{j, i} &= 1, P(i, j)_{j, j} = 0 			
			\end{align*}
			\begin{itemize}[noitemsep]
				\item $P^{-1} = P^T = P \rightarrow $ orthogonal
				\item $det (P) = -1$
				\item Multiplying $A$ with $P$ swaps rows/cols
			\end{itemize}
		\end{multicols}
		\item \begin{multicols}{2}
			$N(\alpha; i \rightarrow j)$ differs from $I_n$ in 1 elements \begin{align*}
			N(\alpha; i \rightarrow j)_{j, i} = \alpha			
			\end{align*}
			\begin{itemize}[noitemsep]
				\item $N(\alpha; i \rightarrow j)^{-1} = N(-\alpha; i \rightarrow j) \rightarrow N^{-1} \neq N^T $ non orthogonal
				\item $det N = 1$
				\item Multiplying $A$ with $N$ adds the $\alpha$-fold of row/col $i$ to row/col $j$
			\end{itemize}
		\end{multicols}
	\end{enumerate}

	\textcolor{red}{$S, N$ are not orthogonal $\rightarrow$ the condition number of the system changes}
\end{definition}

\begin{definition}[Summary - Classical Gauß]
	$A$ is reshaped by non-singular matrices $Q$ that are multiplied from the left which only considers row operations:
	\begin{align*}
		Ax=b \leftrightarrow (QA)x = Qb \leftrightarrow (Q^{-1}QA)x = (Q^{-1}Q)b \leftrightarrow Ax = b \text{\; no loss of information} \\
		det (QA) = det(Q) \cdot det(A) \text{\; determinant bypass of elementary transformation matrices}
	\end{align*}
\end{definition}

\begin{definition}[Gauß with columns]
	We can use the same transformations for column operations - however, they solve a different system:
	\begin{align*}
	Ax&=b \leftrightarrow A(Q^{-1}Q)x=b \leftrightarrow (AQ^{-1})(Qx)=b \leftrightarrow AQy = b \\
	y &= Q^{-1}x  \leftrightarrow x = Qy \text{\; substitution at the end}
	\end{align*}
\end{definition}

\begin{satz}[Gauß with Pivoting]
	Gauß with pivoting preserves symmetry and positive definiteness in the residual matrix. I.e., for pos./neg. definite matrices Gaussian elimination is possible without pivoting
\end{satz}

\begin{definition}[LU decomposition/factorization]
	Decompose Matrix into $A = LU$ where $L$ Lower unit triangular matrix, $U$ upper triangular matrix. Gauß and LU deliver the same result with same operations (excluding permutations). Ansatz
	\begin{align*}
		U = Q_1 \cdot ... \cdot Q_n A \rightarrow \underbrace{\inv{Q_n} \cdot ... \cdot \inv{Q_1}}_{=: L} U = A
	\end{align*}
	The inverse of the elementary transformation matrices is thereby easily to obtain. If permutation operations $P$ occur:
	\begin{itemize}[noitemsep]
		\item perform then at the beginning of a principal step
		\item multiply old $L^{(1)}$ on both sides with $P$ yielding $\tilde{L}$
		\item now multiply $\tilde{L}$ with $L^{(2)}$ giving $\hat{L}^{(2)}$ and $PA = \hat{L}^{(2)}A^{(2)}$
	\end{itemize}

	Comparison:
	
	\begin{multicols}{2}
		Gauß: always $\BigO{n^3}$
		\begin{align*}
		 \text{simulataneously}	\begin{cases}
				U = \inv{L}A \\
				y = \inv{L}b 
			\end{cases} \\
			Ux = y \medspace \medspace \text{Backward Substitution}
		\end{align*}
		LU: initially $\BigO{n^3}$, reuse: $\BigO{n^2}$
		\begin{align*}
			A &= LU \\
			Ly &= b \medspace \medspace \text{Forward Substitution}\\
			Ux &= y \medspace \medspace \text{Backward Substitution}
		\end{align*}
	\end{multicols}
\end{definition}

\begin{definition}[Iterative Refinement]
	Increase precision end the end (need double precision for $z$)
	\begin{enumerate}[noitemsep]
		\item Compute Residual $r = b - Ax^{(i)}$
		\item Solve system $Az = r$
		\item Update Solution $x^{(i+1)} = x^{(i)} + z$
		\item Stop if $\frac{\norm{z}}{\norm{x^{(i+1)}}} < tol$
	\end{enumerate}
\end{definition}

\begin{definition}[Gauß for general $n \times n$ matrices]
	content...
\end{definition}

\pagebreak

\section{Error Analysis}


\section{Least Squares}


\section{Interpolation}


\section{Quadrature}

\section{Eigenvalues and Eigenvectors}


\section{Non-Linear Equations}


\section{Iterative Solution of Linear Systems}

\section{Numerical Treatment of ODEs}
\begin{itemize}
	\item linear, nonlinear, homogen, inhomogeneous
	\item analytical solutions
	\item Population models, oscillators
	\item Stable solutions
	\item Direction field, phase plots
\end{itemize}


\section{Finite Difference Methods}

\begin{itemize}
	\item time-dependent PDE: we have a dependency on the time derivative: $\frac{\partial f}{\partial t} = ...$
	\item stationary PDE: PDE is not dependent on time derivative but to a function 
	\item equilibrium solution of stationary PDE: function is equal to 0
	\item initial value problem: time-dependent differential equation with given $u(t=0)$
	\item boundary value problem: no time dependency, values on boundaries given
	\item classes of boundary conditions: Dirichlet/Neumann/Robin/Mixed
	\item initial boundary problem: both starting value and values at boundaries given
	\item spacing, mesh size: height(s) of grid discretizing domain
	\item Heat equation vs. poisson equation
	\item Laplace operator
\end{itemize}

\begin{itemize}
	\item forward difference
	\item backward difference
	\item central difference 
	\item stiffness matrix
	\item implicit/explicit methods
	\item visualization: stepping vs integration
	\item stability of an differential equation
\end{itemize}

Derivatives and physical/engineering  terms: 

\begin{tabular}{c|c|c}
	$f$ & $f'$ & $f''$ \\
	\hline
	Population & Growth & - \\ 
	\hline
	Position & Velocity & Acceleration \\
	\hline
	Position & Displacement & Curvature
\end{tabular}

\section{Finite Element Method}

\section{Finite Volume Method}


