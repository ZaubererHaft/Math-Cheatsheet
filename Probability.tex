\section{Allgemeines}
\begin{definition}[Wahrscheinlichkeitsraum]
	$W=(\Omega, \sigma, Pr)$ heißt Wahrscheinlichkeitsraum falls
	\begin{itemize}[noitemsep]
		\item $\Omega$ beliebige Menge, genannt \emph{Ereignismenge}, $x \in \Omega$ heißt Elementarereignis, $B \subseteq \Omega$ nennt man Ereignis
		\item $\sigma$ Sigma-Algebra über $\Omega$. $A$ heißt Sigma Algebra über $\Omega$, falls
		\begin{enumerate}[noitemsep]
			\item $\Omega \in A$
			\item $B \in  A \rightarrow \bar{B} \in A$
			\item $B_1, B_2, \dots \in A \rightarrow \bigcup_{i = 1}^\infty B_i \in A$
		\end{enumerate}
	\item $Pr$ Wahrscheinlichkeitsmaß über $\sigma$, d.h.
	\begin{enumerate}[noitemsep]
		\item $\pr{\Omega} = 1$
		\item $B_1, B_2$ paarweise disjunkt $\rightarrow \pr{\bigcup_{i = 1}^\infty B_i} = \sum_{i=1}^\infty \pr{B_i}$ (Additionssatz)
	\end{enumerate}
	\end{itemize}
	Aus dieser Definition folgt insbesondere:
	\begin{itemize}[noitemsep]
		\item $\pr{\emptyset} = 0$
		\item $\pr{B_i} \geq 0, \pr{B_i} \leq 1 $
		\item $\pr{B_i} = 1 - \pr{\bar{B_i}}$
		\item $A \subseteq B \rightarrow \pr{A} \leq \pr{B}$
	\end{itemize}
\end{definition}

\begin{satz}[Siebformel]
	Für paarweise nicht-disjunkte Ereignisse $B_i$ gilt
	\begin{align*}
		Pr[\bigcup_{i=1}^m B_i] \sum_{k=1}^n ((-1)^{k + 1} \sum_{I \subseteq \{ 1, \dots, n \}, |I| = k } Pr[\bigcap_{i \in I} B_i])
	\end{align*}
	Für $n=3$ gilt z.B. der Speziallfall
	\begin{align*}
		Pr[B_2 \cup B_2 \cup B_3 ] = & Pr[B_1] + Pr[B_2] + Pr[B_2] - \\ & Pr[B_1 \cap B_2] - Pr[B_2 \cap B_3] - Pr[B_1 \cap B_3] + Pr[B_1 \cap B_2 \cap B_3]
	\end{align*}
\end{satz}

\begin{satz}[Bedingte Wahrscheinlichkeit]
	Für Ereignisse $A,B$ gilt:
	\begin{align*}
		Pr[A | B] = \frac{Pr[A \cap B]}{Pr[B]} \rightarrow Pr[A \cap B] = Pr [A | B] \cdot Pr[B]
	\end{align*}
\end{satz}

\begin{satz}[Multiplikationssatz]
	Für Ereignisse $A_1, \cdots, A_n$ gilt:

\end{satz}


\pagebreak
	
\section{Diskret}

\pagebreak

\section{Kontinuierlich}

\pagebreak

\section{Statistik}

\pagebreak

\section{Stochastik}

\pagebreak