\section{Allgemeines}
\begin{definition}[Wahrscheinlichkeitsraum]
	$W=(\Omega, \sigma, Pr)$ heißt Wahrscheinlichkeitsraum falls
	\begin{itemize}[noitemsep]
		\item $\Omega$ beliebige Menge, genannt \emph{Ereignismenge}, $x \in \Omega$ heißt Elementarereignis, $B \subseteq \Omega$ nennt man Ereignis
		\item $\sigma$ Sigma-Algebra über $\Omega$. $A$ heißt Sigma Algebra über $\Omega$, falls
		\begin{enumerate}[noitemsep]
			\item $\Omega \in A$
			\item $B \in  A \rightarrow \bar{B} \in A$
			\item $B_1, B_2, \dots \in A \rightarrow \bigcup_{i = 1}^\infty B_i \in A$
		\end{enumerate}
	\item $Pr$ Wahrscheinlichekeitsmaß über $\sigma$, d.h.
	\begin{enumerate}[noitemsep]
		\item $\pr{\Omega} = 1$
		\item $B_1, B_2$ paarweise disjunkt $\rightarrow \pr{\bigcup_{i = 1}^\infty B_i} = \sum_{i=1}^\infty \pr{B_i}$ (Additionssatz)
	\end{enumerate}
	\end{itemize}
	Aus dieser Definition folgt insbesondere:
	\begin{itemize}[noitemsep]
		\item $\pr{\emptyset} = 0$
		\item $\pr{B_i} \geq 0, \pr{B_i} \leq 1 $
		\item $\pr{B_i} = 1 - \pr{\bar{B_i}}$
		\item $C \subseteq B \rightarrow \pr{A} \leq \pr{B}$
	\end{itemize}
\end{definition}


	
\section{Diskret}

\section{Kontinuierlich}

\section{Statistik}

\section{Stochastik}

\pagebreak