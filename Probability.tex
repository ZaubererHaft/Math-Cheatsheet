\section{Allgemeines}
\begin{definition}[Wahrscheinlichkeitsraum]
	$W=(\Omega, \sigma, Pr)$ heißt Wahrscheinlichkeitsraum falls
	\begin{itemize}[noitemsep]
		\item $\Omega$ beliebige Menge, genannt \emph{Ereignismenge}, $x \in \Omega$ heißt Elementarereignis, $B \subseteq \Omega$ nennt man Ereignis
		\item $\sigma$ Sigma-Algebra über $\Omega$. $A$ heißt Sigma Algebra über $\Omega$, falls
		\begin{enumerate}[noitemsep]
			\item $\Omega \in A$
			\item $B \in  A \rightarrow \bar{B} \in A$
			\item $B_1, B_2, \dots \in A \rightarrow \bigcup_{i = 1}^\infty B_i \in A$
		\end{enumerate}
	\item $Pr$ Wahrscheinlichkeitsmaß über $\sigma$, d.h.
	\begin{enumerate}[noitemsep]
		\item $\pr{\Omega} = 1$
		\item $B_1, B_2$ paarweise disjunkt $\rightarrow \pr{\bigcup_{i = 1}^\infty B_i} = \sum_{i=1}^\infty \pr{B_i}$ (Additionssatz)
	\end{enumerate}
	\end{itemize}
	Aus dieser Definition folgt insbesondere:
	\begin{itemize}[noitemsep]
		\item $\pr{\emptyset} = 0$
		\item $\pr{B_i} \geq 0, \pr{B_i} \leq 1 $
		\item $\pr{B_i} = 1 - \pr{\bar{B_i}}$
		\item $A \subseteq B \rightarrow \pr{A} \leq \pr{B}$
	\end{itemize}
\end{definition}

\begin{satz}[Siebformel]
	Für paarweise nicht-disjunkte Ereignisse $B_i$ gilt
	\begin{align*}
		Pr[\bigcup_{i=1}^m B_i] \sum_{k=1}^n ((-1)^{k + 1} \sum_{I \subseteq \{ 1, \dots, n \}, |I| = k } Pr[\bigcap_{i \in I} B_i])
	\end{align*}
	Für $n=3$ gilt z.B. der Speziallfall
	\begin{align*}
		Pr[B_2 \cup B_2 \cup B_3 ] = & Pr[B_1] + Pr[B_2] + Pr[B_2] - \\ & Pr[B_1 \cap B_2] - Pr[B_2 \cap B_3] - Pr[B_1 \cap B_3] + Pr[B_1 \cap B_2 \cap B_3]
	\end{align*}
\end{satz}

\begin{satz}[Bedingte Wahrscheinlichkeit]
	Für Ereignisse $A,B$ gilt:
	\begin{align*}
		Pr[A | B] = \frac{Pr[A \cap B]}{Pr[B]} \rightarrow Pr[A \cap B] = Pr [A | B] \cdot Pr[B]
	\end{align*}
\end{satz}

\begin{satz}[Multiplikationssatz]
	Für Ereignisse $A_1, \dots, A_n, Pr[A_1 \cap, \dots ,\cap A_n] > 0$ gilt:
	\begin{align*}
		Pr[A_1 \cap, \dots ,\cap A_n] = Pr[A_1]  \cdot Pr[A_2 | A_1] \cdot Pr[A_2|A_1] \cdot Pr[A_3 | A_1 \cap A_2] \cdot  ... \\ \cdot Pr[A_n | A_1 \cap \dots A_n] 
	\end{align*}
\end{satz}

\begin{satz}[Satz der totalen Wahrscheinlichkeit]
	Seien $A_1,\dots,A_n$ paarweise disjunkt, $Pr[A_i] > 0, \forall i$. Sei $B \subseteq A_1 \cup \dots \cup A_n$. Dann gilt:
	\begin{align*}
	Pr[B] = \sum_{i=1}^{n} Pr[A_i \cap B] = \sum_{i=1}^n Pr[B | A_i] \cdot A_i
	\end{align*}
	Gilt zusätzlich $Pr[B] > 0$ (\textbf{Satz von Bayes}):
	\begin{align*}
		Pr[A_i | B] = \frac{Pr[A_i \cap B]}{Pr[B]} = \frac{Pr[B | A_i] \cdot Pr[A_i]}{\sum_{j=1}^n Pr[B | A_j] \cdot Pr[A_j]}
	\end{align*}
\end{satz}

\begin{definition}[Unabhängigkeit von Ereignissen]
	Ereignisse $A, B$ heißen unabhjängig wenn $\prob{A \cap B} = \prob{A} \cdot \prob{B}$
\end{definition}

\begin{definition}[Zufallsvariable]
	Eine Zufallsvariable ist eine Abbildung $\function{X}{\Omega}{\realset}$. $W_X := X(\Omega)$ ist die Bildmenge oder Wertemenge. Zufallsvariablen sind also im Wesentlichen eine Kurzschreibweise für bestimmte Ereignisse.
	\begin{itemize}[noitemsep]
		\item Wir schreiben $\prob{X = x_i}$ für das Ereignis $\prob{\inv{X}(x_i)} = \prob{\{A \medspace | \medspace X(A) = x_i\}}$
		\item $f_X(x) := \prob{X = x}$ heißt Dichtefunktion
		\item $F_X(x_i) := \prob{X \leq x_i} = \sum_{x \in W_X, x \leq x_i} \prob{X = x}$ heißt Verteilungsfunktion
	\end{itemize} 
\end{definition}

\begin{definition}[Unabhängigkeit von Zufallsvariablen]
	Zwei Zufallsvariablen $X,Y$ heißen unabhängig, wenn $f_{X,Y}(x,y) = f_X(x) \cdot f_Y(y) \leftrightarrow F_{X,Y}(x,y) = F_X(x) \cdot F_Y(y)$
\end{definition}

\begin{definition}[Randdichte zweier Zufallsvariablen]
	Seien $X,Y$ Zufallsvariablen.
	$f_X(x) = \sum_{y \in W_Y} f_{X,Y}(x,y)$
\end{definition}

\begin{definition}[Randverteilung zweier Zufallsvariablen]
	Seien $X,Y$ Zufallsvariablen.
	$F_X(x) = \sum_{x' \leq X} f_X(x') = \sum_{x' \leq X} \sum_{y \in W_Y} f_{X,Y}(x',y)$
	wobei
	\begin{itemize}[noitemsep]
		\item $f_{X,Y}(x,y) = \pr{X = x, Y = y}$ gemeinsame Dichte
		\item $F_{X,Y}(x,y) = \pr{X \leq x, Y \leq y}$ gemeinsame Verteilung
	\end{itemize}
\end{definition}

\begin{definition}[Faltung (zusammengesetzte Zufallsvariablen)]
	Seien $Z = X + Y$, $X,Y$ unabhängig.
	$f_Z(z) = \sum_{x \in W_X} f_X(x)f_Y(z-x)$
\end{definition}


\begin{definition}[Abschätzen von Erwartungswert und Varianz]
	Für kontinuierliche bzw. diskrete Erwartungswerte und Varianz gilt
	$\expec{Z^2} = \var{Z} + \expec{Z}^2 \geq \expec{Z}^2$
\end{definition}

\pagebreak
	
\section{Diskret}

\begin{definition}[Erwartungswert]
	Der Erwartungswert ist der Mittelwert, wenn  ein Zufallsexperiment unendlich oft wiederholt wird. Er gibt an, mit welchem Wert man auf lange Sicht bei einem Zufallsexperiment rechnen kann.
	Der diskreter Erwartungswert ist definiert als
	\begin{align*}
		\expec{X} = \mu := \sum_{x \in W_X} x \cdot \pr{X = x}
	\end{align*}
	Eigenschaften:
	\begin{itemize}[noitemsep]
		\item $\sum_{x \in W_X} |x| \cdot \pr{X = x}$ konvergiert
		\item Linearität: $\expec{a_1x_1 + ... + a_nx_n} = a_1\expec{x_1} + ... + a_n \expec{x_n}$
		\item Multiplikativität: $\suboneton{X}$ unabhängig $\rightarrow \expec{X_1 \cdot ... \cdot X_n} = \expec{X_1} \cdot ... \cdot \expec{X_n} $ 
		\item Monotonie: $\forallin{\omega}{\Omega} : X(\omega) \leq Y(\omega) \rightarrow \expec{X} \leq \expec{Y}$
		\item Totalität: $A_i$ paarweise disjunkt mit $A_1 \cup ... \cup A_n = \Omega, \pr{A_i} > 0$. $\rightarrow \expec{X} = \sum_{i = 1}^n \expec{X | A_i} \cdot \pr{A_i}$
	\end{itemize}
\end{definition}


\begin{definition}[Varianz und Standardabweichung]
	Die Varianz ist ein Streuungsparameter und bezeichnet die mittlere quadratische Abweichung vom \\ Erwartungswert. Die diskrete Varianz ist definiert als
	\begin{align*}
	\var{X} = \sigma^2 := \expec{(X - \mu)^2} = \sum_{x \in W_x} (x - \mu )^2 \cdot \pr{X = x}
	\end{align*}.
	Die Größe $\sigma = \sqrt{\var{X}}$ heißt Standardabweichung von $X$.
	
	Eigenschaften:
	\begin{itemize}[noitemsep]
		\item Additivität: $X_1, ..., X_n$ unabhängig $\rightarrow \var{X_1 + ... + X_n} = \var{X_1} + ... + \var{X_n}$
		\item Linearität: $\var{aX + b} = a^2 \cdot \var{X}$
	\end{itemize}
\end{definition}

\begin{definition}[Bernoulli Verteilung]
	Sei $X$ eine Zufallsvariable mit $W_X = \{0, 1\}$.
	\begin{align*}
		f_X(x) = \begin{cases} p &, x = 1 \\ 1 - p &, x = 0\end{cases}, \medspace \expec{X} = p, \medspace \var{X} = p \cdot (1 - p)
	\end{align*}
\end{definition}

\begin{definition}[Binomial Verteilung]
	Sei $X$ eine Zufallsvariable mit $X=X_1 + ... + X_n$ Summe $n$ unabhängiger Bernoulli verteilten Zufallsvariablen.
	\begin{align*}
	f_X(x) = \binom{n}{x} p^x (1-p)^{n-x}, \medspace \expec{X} = np, \medspace \var{X} = np(1-p)
	\end{align*}
	Wir schreiben $X \sim Bin(n,p)$. Für $Z = X + Y, X \sim Bin(nx, p), Y \sim Bin(ny, p)$ gilt $Z \sim Bin(nx + ny, p)$ 
\end{definition}

\begin{definition}[Geometrische Verteilung]
	Die geometrische Verteilung drückt die Wahrscheinlichkeit aus, dass $i$ Bernoulli erfolglos sind, aber der $i+1$ erfolgreich ist. 
	\begin{align*}
	f_X(i) &= p(1-p)^i, i \in \naturalset, \expec{X} = \frac{1}{p}, \var{X} = \frac{1 - p}{p^2} \\
	&\rightarrow \pr{X > i} = (1-p)^i \\
	&\rightarrow \expec{X - i | X > i} = \expec{X} \\
	&\rightarrow \pr{X > y + x | X > x} = \pr{X > y}
	\end{align*}
\end{definition}

\begin{definition}[Poisson Verteilung]
	Bei der Poisson Verteilung ist bekannt, dass im Mittel $\lambda$ viele auftreten. Wie hoch ist die Wahrscheinlichkeit, dass $i$ auftreten? Wir schreiben $X \sim Po(\lambda)$.
	\begin{align*}
		f_X(i) = \frac{e^{-\lambda \cdot \lambda^i}}{i!}, \medspace \expec{X} = \lambda, \medspace \var{X} = \lambda
	\end{align*}
	Es gilt: $X \sim Bin(n,p) \overset{n \rightarrow \infty}{\rightarrow} Po(\lambda)$
\end{definition}

\begin{satz}[Abschätzen von Wahrscheinlichkeiten]
	Zum Abschätzen von \\ Wahrscheinlichkeiten gelten die folgenden Ungleichungen
	\begin{itemize}[noitemsep]
		\item Markov: $\pr{X \geq t} \leq \frac{\expec{X}}{t}$
		\item Chebychev: $\pr{|X - \expec{X}| \geq t} \leq \frac{\var{X}}{t^2}$
		\item Chernoff: $X \sim Bin, \mu := \expec{X}, \delta > 0$.
			\begin{align*}
				&\rightarrow \pr{X \geq (1 + \delta) \mu} \leq (\frac{e^\sigma}{(1+\delta)^{1+\delta}})^\mu \\
				&\rightarrow \pr{X \leq (1 + \delta) \mu} \leq (\frac{e^\sigma}{(1-\delta)^{1-\delta}})^\mu, \delta \in (0,1)
			\end{align*}
	\end{itemize}
\end{satz}

\begin{definition}[Wahrscheinlichkeitserzeugende Funktion]
	Die Wahrscheinlichkeitserzeugende Funktion zu einer Zufallsvariable $X$ mit $W_X \subseteq \naturalset_0$ berechnet die Momente einer Zufallsvariable 
	
	\begin{align*}
		G_X(s) &= \sum_{x \in W_X} \pr{X = k} \cdot x \overset{W_X \subseteq \naturalset_0}{=} \sum_{k=0}^{\infty}\pr{X = k} \cdot s^k =: \expec{s^X} \\
		\text{Folgerungen:} \\
	    \rightarrow G_X(1) &= 1, \rightarrow G'_X(1) = \expec{X} \\
	    \rightarrow G''_X(1) &+ G'_X(1) - G'_X(1)^2 = \var{X}
	\end{align*}
\end{definition}

\begin{definition}[Momenterzeugende Funktion]
	Momenterzeugende Funktionen sind definiert als
	\begin{align*}
		M_X(s) := \expec{e^{Xs}} = G_X(e^s)
	\end{align*}
\end{definition}

\pagebreak

\begin{satz}[Wichtige wahrscheinlichkeitserzeugende Funktionen]
	Zu bekannten Verteilungen lassen sich die folgenden wahrscheinlichkeitserzeugende Funktionen finden:
	\begin{itemize}[noitemsep]
		\item Bernoulli: $G_X(s) = 1 - p + ps$		
		\item Binomial:  $G_X(s) = (1 - p + ps)^n$
		\item Geometrisch: $G_X(s) = \frac{ps}{1 - (1 - p)^s}$
		\item Poisson: $G_X(s) = e^{\lambda(s - 1)}$
	\end{itemize}
\end{satz}

\begin{satz}[Erzeugende Funktion einer Summe]
	Seien $\suboneton{X}$ unabhängig, $Z := X_1 + ... + X_n$. Dann gilt:
	\begin{align*}
		G_Z(s) &= G_{X_1}(s) \cdot ... \cdot G_{X_n}(s) \\		
		M_Z(s) &= M_{X_1}(s) \cdot ... \cdot M_{X_n}(s) \\
	\end{align*}
\end{satz}

\begin{definition} [Hypergeometrische Verteilung]
	Sei $a+b$ die Grundgesamtheit, $b$ die für das Ergebnis günstigen. Sei $r$ die Größe der entnommenen Stichprobe, $x$ die Anzahl der Elemente in $b$. Dann gilt
	
	\begin{align*}
	 \pr{X = x} =	\frac{\binom{b}{x} \binom{a}{r-x} }{\binom{a + b}{r}}
	\end{align*}
	
\end{definition}

\pagebreak

\section{Kontinuierlich}

\begin{definition}[Kontinuierliche Zufallsvariable]
	$X$ ist eine kontinuierliche Zufallsvariable mit Dichte $f_X$, falls $\int_{- \infty}^{+ \infty} f_X(x) dx = 1$.
    Ein Ereignis $A$ tritt dabei ein, wenn $X$ einen Wert aus $A$ annimmt:
    $\pr{A} = \int_A f_X(x) dx$
\end{definition}

\begin{definition}[Zusammenhan Dichte und Verteilung]
	Für die kontinuierliche Verteilungsfunktion gilt: $F_X(x) = \int_{- \infty}^x f_X(t)dt$. Sie muss:
	\begin{itemize} [noitemsep]
		\item monoton und stetig sein sowie
		\item $\lim_{x \rightarrow -\infty}F_X(x) = 0, \lim_{x \rightarrow \infty}F_X(x) = 1$
	\end{itemize}
	Dann ist $f_X(x) = F'_X(x)$
\end{definition}

\begin{definition}[Kontinuierlicher Erwartungswert]
	Der kontinuierliche \\ Erwartungswert ist definiert als 
	\begin{align*}
		\expec{X} := \int_{- \infty}^{+ \infty} t \cdot f_X(t) dt
	\end{align*}
	Er existiert dabei, falls $ \int_{- \infty}^{+ \infty} |t| \cdot f_X(t) dt$ konvergiert. Es gelten die gleichen Regeln wie für den diskreten Erwartungswert.
\end{definition}

\begin{definition}[Kontinuierliche Varianz]
	Die kontinuierliche Varianz ist definiert als 
	\begin{align*}
	\var{X}:= \int_{- \infty}^{+ \infty} (t - \expec{X}) \cdot f_X(t) dt = \expec{X^2} - \expec{X}^2 = \expec{(X - \expec{X})^2}
	\end{align*}
	Es gelten die gleichen Regeln wie für die diskrete Varianz.
\end{definition}	

\begin{definition}[Gemeinsame Dichte/Verteilung]
	Seien $X,Y$ kontinuierliche Zufallsvariablen. Der gemeinsame Wahrscheinlichkeitsraum über $\realset^2$ ist durch die gemeinsame Dichtefunktion $f_{X,Y}$ beschrieben und es gilt
	
	\begin{align*}
		\int_{-\infty}^\infty \int_{-\infty}^\infty f_{X,Y}(x,y)dx dy = 1 \\
		\rightarrow \prob{A} = \int_A f_{X,Y}(x,y) dx dy
	\end{align*}
	
	Die gemeinsame Verteilung ist definiert als
	
	\begin{align*}
		F_{X,Y}(x,y) = \pr{X \leq x, Y \leq y} = \int_{- \infty}^y \int_{- \infty}^y f_{X,Y}(u,v) du dv
	\end{align*}
	Und man definiert die
	\begin{align*}
		\text{Randdichte} \medspace f_X(x) &= \int_{- \infty}^\infty f_{X,Y}(x,v) dv \\
		\text{Randverteilung} \medspace F_X(x) &= \int_{- \infty}^x \int_{-\infty}^{\infty} f_{X,Y}(u,v) du dv	
	\end{align*}
\end{definition}

\begin{definition}[Gleichverteilung]
	Die Gleichverteilung einer Zufallsvariable über $[a,b]$ ist definiert als:
	\begin{align*}
		f_X(x) = \begin{cases}
			\frac{1}{b - a} & x \in [a,b] \\
			0 & sonst
		\end{cases}, 
		F_{X} = \begin{cases}
		0 & x < a \\
		\frac{x - a}{b - a} & x \in [a,b] \\
		1 & x > b
		\end{cases}, \expec{X} = \frac{a + b}{2}, \var{X} = \frac{(a-b)^2}{12}
	\end{align*}
\end{definition}


\begin{definition}[Normalverteilung]
	Die Normalverteilung einer Zufallsvariable $X$ mit Erwartungswert $\expec{X} =: \mu$ und Varianz $\var{X} =: \sigma^2$ kann als Binominalverteilung für $n \rightarrow \infty$ interpretiert werden. Wir schreiben $X \sim N(\mu, \sigma^2)$
	
	\begin{align*}
	f_X(x) &= \frac{1}{\sqrt{2\pi}\sigma} \cdot e^{-\frac{(x-\mu)^2}{2\sigma^2}}  =: \phi(x;\mu,\sigma) \overset{\mu = 0, \sigma^2 = 1}{=:} \phi(x) \\
	F_X(x) &= \int_{-\infty}^{x} \frac{1}{\sqrt{2\pi}\sigma} \cdot e^{-\frac{(t- \mu)^2}{2 \sigma^2}} dt =: \Phi(x; \mu, \sigma) \overset{\mu = 0, \sigma^2 = 1}{=:} \Phi(x)
	\end{align*}
\end{definition}

\begin{satz}[Normierung der Normalverteilung]
	Sei $X \sim N(\mu, \sigma^2)$. Dann gilt $Y := \frac{X - \mu}{\sigma} \sim N(0,1)$
\end{satz}

\begin{satz}[Lineare Transformation der Normalverteilung]
	Sei $X \sim N(\mu, \sigma^2)$. Dann gilt für $a \in \realset \backslash \{0\}$ und $b \in \realset$: $Y = aX + b \sim N(a\mu + b, a^2 \sigma^2)$ ist ebenfalls normalverteilt.
\end{satz}

\begin{satz}[Additivität der Normalverteilung]
	Für $\suboneton{X}$ unabhängig mit $\sim N(\mu_i,\sigma_i^2)$. Dann ist $Z := a_1N_1 + ... + a_nX_n$  ebenfalls normalverteilt, $Z \sim N(\mu, \sigma^2)$, mit $\mu = \sum_{i=1}^na_i\mu_i$ und $\sigma^2 = \sum_{i=1}^{n} a_i^2\sigma_i^2$        
\end{satz}

\begin{definition}[Exponentialverteilung]
	Die Exponentialverteilung einer Zufallsvariable $X$ kann als Grenzwert der geometrischen Verteilung aufgefasst werden. Wir schreiben $X \sim Exp(\lambda)$ und es gilt:
	
	\begin{align*}
		f_X(x) = \begin{cases}
			\lambda e^{-\lambda x} &, x \geq 0 \\
			0 & sonst
		\end{cases}, F_X(x) = 1 - e^{-\lambda x}, \expec{X} = \frac{1}{\lambda}, \var{\frac{1}{\lambda^2}}
	\end{align*}
	 
\end{definition}

\begin{satz}[Eigenschaften der Exponentialverteilung] Die Exponentialverteilung hat eine Reihe wichtiger Eigenschaften
	\begin{itemize}[noitemsep]
		\item Für $X \sim Exp(\lambda_x)$, $Y \sim Exp(\lambda_y)$, $X,Y$ unabhängig gilt: $Z := X + Y \sim Exp(\lambda_x + \lambda_y)$
		\item $X \sim Exp(\lambda) \rightarrow Y := aX \sim Exp(\frac{\lambda}{a})$ 
		\item Gedächtnislosigkeit: $\pr{X > x + y | X > y} = \pr{X > x}$
	\end{itemize}
\end{satz}

\begin{satz}[Faltungssatz]
	Seien $X,Y$ unabhängig, $Z := X + Y$. $f_Z(z) = \int_{- \infty}^{\infty} f_X(x) \cdot F_Y(z-x) dx $
\end{satz}

\begin{satz}[Zentraler Grenzwertsatz]
	Seien $\suboneton{X}$ unabhängig mit selber Verteilung, Existenz des Erwartungswerts und der Varianz sei gegeben. $Y_n := \sum_{i=1}^{n}X_i$. Dann gilt:
	\begin{align*}
		Z := \frac{Y_n - n \mu}{\sigma \sqrt{n}} \sim N(0,1) \medspace \text{für} \medspace n \rightarrow \infty
	\end{align*}
\end{satz}

\begin{satz}[Grenzwertsatz De Moivre]
	Seien $\suboneton{X}$ unabhängige Bernoulli-verteilte Zufallsvariable mit gleicher Wahrscheinlichkeit $p$. $H := \sum_{i = 1}^n X_i$.Dann gilt:
	\begin{align*}
	Z := \frac{H_n - np}{\sqrt{np(1-p)}} \sim N(0,1)\medspace \text{für} \medspace n \rightarrow \infty
	\end{align*}
\end{satz}

\begin{satz}[Momenterzeugende Funktion]
	Eine Momenterzeugende Funktion ist definiert als $M_X(s) = \expec{e^{Xs}}.$ Es gilt: $M_X(0)^k = \expec{X^k}$. Sie ist definiert für
	\begin{itemize}[noitemsep]
		\item $X \sim Unif(a,b) \rightarrow M_X(s) = \frac{e^{sb} - e^{sa}}{s(b-a)}$
		\item $X \sim N(0,1) \rightarrow M_X(s) = e^{\frac{s^2}{2}}$
		\item $X \sim N(\mu,\sigma^2) \rightarrow M_X(s) = e^{s\mu + \frac{(s \sigma)^2}{2}} $
		\item $X \sim Exp(\lambda) \rightarrow M_X(s) = \frac{\lambda}{\lambda - s} ,$ falls $s < \lambda$
	\end{itemize}
\end{satz}

\pagebreak

\section{Stochastik}

\begin{definition}[Stochastischer Prozess]
	content...
\end{definition}

\pagebreak

\section{Deskriptive und Induktive Statistik}

\pagebreak
